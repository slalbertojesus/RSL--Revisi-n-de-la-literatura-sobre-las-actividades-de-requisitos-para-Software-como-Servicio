\documentclass{article}

\usepackage{enumerate}

\begin{document}
  \title{%
  Protocolo \\
  \large Revisión de la literatura sobre las actividades de requisitos para Software como Servicio\\}
  \author{Alberto de Jesús Sánchez López \\ 
  \small Proyecto Guiado}
  \date{Fecha}
  \maketitle
  \thispagestyle{empty}
  \newpage

  \tableofcontents
  \thispagestyle{empty}
  \newpage

\setcounter{page}{1}
\section{Introducción}
Llevar a cabo un proceso de desarrollo de software que resulte en un producto de calidad, es 
un proceso complejo, que utiliza metodologías y estrategias adecuadas para el análisis del problema 
en cuestión asì cómo el diseño de una solución, para el Software como Servicio (SaaS), esta no es la excepción. 
El Software como Servicio tiene necesidades específicas que evitan la adaptación de metodologías y estrategias 
tradicionales utilizadas dentro de la Ingeniería de Software, esto representa un conjunto de problemas para aquellos 
interesados en iniciar el desarrollo de un software como servicio.

El proceso de desarrollo de software tradicional conlleva un conjunto de fases con el propósito común de asegurar la entrega 
de un producto con calidad y en tiempo, la fase de requisitos, es crucial para delimitar el alcance del proyecto, analizar, 
documentar y verificar los servicios y restricciones del sistema. Una definición de requisitos que ha sido desarrollada siguiendo una 
metodogìa o estrategia, es de suma importancia para el éxito del proyecto, ya que esto asegura que la especificación de requisitos ha sido 
desarrollada siguiendo un proceso formal y por ende los fundamentos necesarios para el correcto diseño de la solución son confiables.

Es por eso, que es de especial interés para aquellos involucrados en el desarrollo de un proceso que resulte en un producto de software 
con calidad. Una fase crucial para el éxito de un proyecto de software es la definición de requisitos. 
\newpage

\section{Preguntas de investigación}
Con el 

\begin{enumerate}[P 1.-]
  \item\emph{¿Qué técnicas de elicitación se han utilizado para la identificación de requisitos de Software como Servicio? }
  \item\emph{¿Qué técnicas de elicitación se han utilizado para la identificación de requisitos de Software como Servicio? }
  \item\emph{¿Qué técnicas de elicitación se han utilizado para la identificación de requisitos de Software como Servicio? }
  \item\emph{¿Qué técnicas de elicitación se han utilizado para la identificación de requisitos de Software como Servicio? }
  \item\emph{¿Qué técnicas de elicitación se han utilizado para la identificación de requisitos de Software como Servicio? }
  \begin{enumerate}[(a)]
  \item sub numbering of 2nd question.
  \item sub numbering of 2nd question.
  \end{enumerate}
\end{enumerate}

\newpage


\section{Estrategia de búsqueda}
\subsection{Términos de búsqueda}
\subsection{Cadenas de búsqueda}
\subsection{Selección de fuentes}
\section{Selección de estudios primarios}
\subsection{Criterios de selección de estudios primarios}
\subsection{Procedimiento de selección de estudios primarios}
\section{Extracción de los datos}
\section{Estrategia para la síntesis de datos}
\section{Limitaciones}
\section{Informe}
\section{Gestión de la revisión}
\section{Rerefencias}

\end{document}
