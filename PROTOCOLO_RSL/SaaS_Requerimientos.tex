\documentclass{article}

\usepackage{enumerate}
\usepackage{makecell}

\begin{document}
  \title{%
  Protocolo \\
  \large Revisión de la literatura sobre las actividades de requisitos para Software como Servicio\\}
  \author{Alberto de Jesús Sánchez López \\ 
  \small Proyecto Guiado}
  \date{Fecha}
  \maketitle
  \thispagestyle{empty}
  \newpage

  \tableofcontents
  \thispagestyle{empty}
  \newpage

\setcounter{page}{1}
\section{Introducción}
Llevar a cabo un proceso de desarrollo de software que resulte en un producto de calidad, es 
un proceso complejo, que utiliza metodologías y estrategias adecuadas para el análisis del problema 
en cuestión asì cómo el diseño de una solución, para el Software como Servicio (SaaS), esta no es la excepción. 
El Software como Servicio tiene necesidades específicas que evitan la adaptación de metodologías y estrategias 
tradicionales utilizadas dentro de la Ingeniería de Software, esto representa un conjunto de problemas para aquellos 
interesados en iniciar el desarrollo de un software como servicio.

El proceso de desarrollo de software tradicional conlleva un conjunto de fases con el propósito común de asegurar la entrega 
de un producto con calidad y en tiempo, la fase de requisitos, es crucial para delimitar el alcance del proyecto, analizar, 
documentar y verificar los servicios y restricciones del sistema. Una definición de requisitos que ha sido desarrollada siguiendo una 
metodogìa o estrategia, es de suma importancia para el éxito del proyecto, ya que esto asegura que la especificación de requisitos ha sido 
desarrollada siguiendo un proceso formal y por ende los fundamentos necesarios para el correcto diseño de la solución son confiables.

Es por eso, que es de especial interés para aquellos involucrados en el desarrollo de un proceso que resulte en un producto de software 
con calidad. Una fase crucial para el éxito de un proyecto de software es la definición de requisitos. 
\newpage

\section{Preguntas de investigación}
Con el 

\begin{enumerate}[P 1.-]
  \item\emph{¿Qué técnicas de elicitación se han utilizado para la identificación de requisitos de Software como Servicio? }
  \begin{enumerate}[(a)]
  \item \emph{¿Qué fuentes de información son utilizadas?}
  \item \emph{¿Qué técnicas de elicitación se aplican?}
  \item \emph{¿Qué retos se presentan en la elicitación?}
  \end{enumerate}
  
  \item\emph{¿Qué técnicas de análisis se han utilizado para la definición de requisitos de Software como Servicio?}
  \begin{enumerate}[(a)]
  \item \emph{¿Cómo se clasifica la información proveniente de la elicitación?}
  \item \emph{¿Cómo se definen los requisitos a partir de la información organizada?}
  \item \emph{¿Cómo se identifican los requisitos no funcionales?}
  \end{enumerate}

  \item\emph{¿Cómo se define el proceso de validación de los requisitos de un Software como servicio?}
  \begin{enumerate}[(a)]
  \item \emph{¿Qué técnicas se utilizan para la validación de los requisitos? }
  \end{enumerate}

  \item\emph{¿Cómo se gestionan los cambios de los requisitos de un Software como Servicio?}
  \begin{enumerate}[(a)]
  \item \emph{¿Cómo se monitorea el entorno para identificar cambios en las condiciones de los requisitos?}
  \item \emph{¿Cómo se evalúan los cambios en los requisitos?}
  \end{enumerate}
            
  \item\emph{¿Cómo se gestionan los riesgos de los requisitos de un Software como Servicio?}
  \begin{enumerate}[(a)]
  \item \emph{¿Cómo se identifican los riesgos?}
  \item \emph{¿Cómo se evalúan los riesgos?}
  \item \emph{¿Cómo se monitorean los riesgos?}
  \item \emph{¿Cómo se controlan los riesgos?}
  \end{enumerate}
  
  \item\emph{¿Qué temas abiertos se identifican en la literatura reciente en el desarrollo de software como servicio?}
  \begin{enumerate}[(a)]
  \item \emph{¿Qué temas abiertos existen relacionado a métodos para requisitos?}
  \item \emph{¿Qué temas abiertos existen relacionado a herramientas para requisitos?}
  \end{enumerate}

\end{enumerate}

\newpage


\section{Estrategia de búsqueda}


\subsection{Términos de búsqueda}
Los siguientes términos de búsqueda fueron seleccionados con el propósito de identificar los estudios que 
permiten proveer evidencia que es relevante a las preguntas de investigación

\begin{table}[ht]
        \caption{Términos de búsqueda} 
        \centering % used for centering table
        \begin{tabular}{c c}
                \hline
                Concepto & Término de búsqueda\\ [0.5ex] % inserts table
                %heading
                \hline % inserts single horizontal line
                Requisitos             & \makecell{Requirements Engineering \\
                Requirements Elicitation \\
                Requirements Validation \\
                Requirements Analysis \\
                Requirements Management \\
                Requirements Risk Management} \\ 
                \hline % inserts single horizontal line
                Software como servicio & \makecell{Software as a Service \\
                SaaS \\
                Cloud Computing \\
                Cloud Services\\
                Service-oriented Requirements Engineering \\
                SORE } \\[1ex] 
                \hline 
        \end{tabular}
        \label{table:tablaterminos} % is used to refer this table in the text
\end{table}

\subsection{Cadenas de búsqueda}

\subsection{Selección de fuentes}
\section{Selección de estudios primarios}
\subsection{Criterios de selección de estudios primarios}
\subsection{Procedimiento de selección de estudios primarios}
\section{Extracción de los datos}
\section{Estrategia para la síntesis de datos}
\section{Limitaciones}
\section{Informe}
\section{Gestión de la revisión}
\section{Rerefencias}

\end{document}
