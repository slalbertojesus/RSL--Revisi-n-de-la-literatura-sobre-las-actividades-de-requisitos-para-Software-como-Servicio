% !TeX spellcheck = es_ES 
\documentclass{article}

\usepackage[section]{placeins}
\usepackage{enumerate}
\usepackage{makecell} 
\usepackage{graphicx}
\usepackage{url}
\usepackage[spanish]{babel}
\usepackage[utf8]{inputenc}
\usepackage[backend=biber, style=numeric]{biblatex}
\addbibresource{PROTOCOLO_RSL.bib}

\begin{document}
  \title{%
  Protocolo \\
  \large Revisión de la literatura sobre las actividades de requisitos para Software como Servicio\\}
  \author{Alberto de Jesús Sánchez López \\ 
  \small Proyecto Guiado}
  \date{Fecha}
  \maketitle
  \thispagestyle{empty}
  \newpage

  \tableofcontents
  \thispagestyle{empty}
  \newpage

\setcounter{page}{1}
\section{Introducción}
Desarrollar un producto de \emph{software} que pueda ser distribuido en un modelo de \emph{software} como servicio 
es un proceso complejo, porque el  producto debe aportar un conjunto de ventajas funcionales hacia el usuario final, 
éstas ventajas representan un valor competitivo a empresas de alto impacto interesadas en entrar en un mercado. Es importante 
mencionar que no existen procesos logísticos externos, ya que la gestión del producto se lleva a cabo en su totalidad, en línea.
Lo anterior permite olvidarse de problemas relacionados a la gestión interna de las funcionalidades del software, 
esto se traduce en una forma efectiva de mitigar costos operacionales dentro de la empresa. 

Por lo tanto, la creación de un \emph{software} como servicio representa un conjunto de retos, debido a que las 
metodologías y estrategias tradicionales no cubren las necesidades para desarrollar un \emph{Saas}. lo que hace necesario adecuar el conocimiento 
existente \cite{wanderley2017requirements}. El éxito de un \emph{Saas} depende de entender y definir el conjunto de requisitos dictados por cliente o mercado \cite{4960886}, 
la fase de requisitos, es crucial para delimitar el alcance del proyecto, analizar, documentar y verificar los servicios y restricciones del sistema. 
Una definición de requisitos que ha sido desarrollada siguiendo un conjunto de estrategias formales,  
es de suma importancia para el éxito del proyecto \cite{6530271} ya que esto garantiza que la especificación de requisitos ha sido 
realizada siguiendo un proceso y los fundamentos necesarios para el correcto diseño de la solución son confiables,
lo que asegura que existe una documentación formal de las necesidades del sistema.
Sin embargo no existen metodologías tradicionales relacionadas al desarrollo y gestión de requisitos para un \emph{Saas}, no existe evidencia 
formal que ataque el proceso de elicitación y gestión de cambios de requisitos en el dominio de \emph{Software} como servicio \cite{7037656}. 

Para analizar el estado de las investigaciones actuales sobre el tema se llevaron a cabo búsquedas de estudios secundarios 
relacionados al proceso de requisitos en un \emph{software} como servicio o computación en la nube, se encontraron
dos estudos secundarios de interés.
El estudio secundario \cite{zalazar2017analyzing} identifica las metodologías utilizadas para el proceso de requisitos en sistemas de la nube,
clasifica a los stakeholders considerados en el proceso de requisitos y clasifica las dificultades encontradas en publicaciones 
relacionadas a requisitos para cómputo en la nube, la investigación realiza una identificación y clasificación, de metodologías soportadas, de forma general
el acercamiento hacia los roles es puramente organizacional y no se detallan las actividades típicas realizadas por los roles señalados en el estudio. 
En el segundo estudio secundario \cite{wanderley2017requirements} se realizó un análisis de metodologías, modelos o herramientas para abordar requisitos en sistemas en la nube, el estudio 
señala el enfoque principal de investigaciones, el tipo de distribución utilizada en las publicaciones seleccionadas y las fuentes de distribución de 
estudios primarios relacionados al \emph{software} como servicio. 

Los estudios anteriores, no se encargan de cerrar el vacío de conocimiento en el área de requisitos para un software como servicio, ya que ninguno de los dos 
analiza el estado del arte de las actividades de requisitos para un \emph{software} como servicio. Esto representa una oportunidad para realizar una revisión
sistemática de la literatura, con el objetivo de analizar e identificar las actividades relacionadas a requisitos en un \emph{software} como servicio,
Esto permitirá a investigadores y estudiantes obtener una recopilación reciente del conjunto de estrategias utilizadas para definir los fundamentos de un producto, 
así como ofrecer un conjunto de áreas de investigación abierta.

\newpage

\section{Esquema de Fundamentos}
Como parte del conjunto de investigaciones realizadas en la Facultad de Estadística e Informática, con el propósito de
explorar campos de conocimiento relacionados al \emph{software, platform e infrastructure as a service}, se encuentra 
la monografía realizada por \cite{hernandez2020xaas}, que se encargó de realizar una compilación de los textos  
relacionados al modelo de distribución \emph{XaaS} (por su siglas en inglés, todo como servicio). En el trabajo, se propone
mantener actualizado el estado del arte de los modelos de distribución en crecimiento o tecnologías que posibilitan 
las condiciones necesarias para llevar a cabo emprendimiento de alto impacto. 

En el contexto mundial actual, la pandemia producida por el virus SARS-CoV-2 (COVID-19) ha resultado en el incremento de  adopción de servicios \emph{cloud} y 
productos con modelo \emph{as a service}, en sectores enfocados a educación en línea, cuidado de la salud y \emph{e-commerce} \cite{value:online} .
Esto se ha provocado debido a que existen empresas dipuestas a adquirir tecnología con el propósito de reducir costos y ofrecer mejor calidad en los
servicios, con solo rentar el derecho de utilizar un producto o solución de infraestructura \cite{Wu2011}.
El acceso al servicio se renta, con la justificación de encontrar valor competitivo o mejora del rendimiento en el servicio a prestar \cite{oliveira20191}.
En resumen, un \emph{software} como servicio puede ser representado como un conjunto de necesidades complejas, dinámicas y extensas, 
las cuales se traducen en un conjunto de requisitos que deben ser identificados, clarificados, 
gestionados y documentados, por medio de un proceso formal. 

La gestión de los requisitos se divide en un conjunto de actividades; elicitación, análisis, especificación y validación \cite{abran2004software}
ëstas disciplinas, abarcan las tareas involucradas con explorar, evaluar, documentar y validar los requisitos de un producto. \cite[p.15]{2543993}.








\newpage

\section{Preguntas de investigación}\label{pi}
El objetivo de la Revisión Sistemática de la Literatura es encontrar el estado del arte del las actividades de requisitos para un \emph{software} como servicio. 

\begin{enumerate}[P 1.-]
  \item\emph{¿Qué técnicas de elicitación se han utilizado para la identificación de requisitos de Software como Servicio?}
  \begin{enumerate}[(a)]
  \item \emph{¿Qué retos se presentan en la elicitación?}
  \end{enumerate}
  Motivación: Señalar el conjunto de técnicas utilizadas para llevar a cabo un proceso de elicitación de requisitos para un software como servicio e identificar los retos encontrados en el proceso de elicitación. 
  
  \item\emph{¿Qué técnicas de análisis se han utilizado para la definición de requisitos de Software como Servicio?}\\
  Motivación: Identificar las actividades realizadas para llevar a cabo el proceso de análisis, clasificación y definición de un conjunto de requisitos para un software como servicio.

  \item\emph{¿Qué actividades se han utilizado para llevar a cabo la validación de los requisitos de un Software como Servicio?}\\
  Motivación: Identificar las técnicas que utilizadas para definir un proceso de validación de requisitos para un software como servicio.

  \item\emph{¿Qué temas abiertos se identifican en la literatura reciente en el desarrollo de Software como Servicio?}
  \begin{enumerate}[(a)]
  \item \emph{¿Qué temas abiertos existen relacionados a las actividades llevadas a cabo en la gestión de requisitos de un software como servicio?}
  \end{enumerate}
  Motivación: Identificar los temas abiertos sugeridos en la literatura relacionada a las actividades de elicitación, análisis, validación y gestión de cambios 
 para requisitos de un software como servicio.
\end{enumerate}

\newpage 


\section{Estrategia de búsqueda}\label{Estrategia de búsqueda}

\subsection{Términos de búsqueda}
Los siguientes términos de búsqueda fueron seleccionados con el propósito de identificar los estudios que 
permiten proveer evidencia relevante a las preguntas de investigación definidas. 
Para lograr lo antes mencionado, se extrajo un conjunto de palabras clave encontradas en las 
preguntas de investigación y se llevó a cabo un conjunto de búsquedas piloto con el fin de encontrar
un conjunto de términos de búsqueda adecuados para hallar investigaciones primarias relevantes a 
la revisión sistemática de la literatura, también se analizó la lectura de artítuclos especificados en la bibliografía recomendada, 
de tal proceso de extrajo el término \emph{Collaborative Requirements}, ya que las necesidades 
específicas de un mercado son expresadas por un conjunto de usuarios clave y es primordial incluir procesos de gestión de requisitos 
en un conjunto de clientes.
A continuación se muestran los términos seleccionados.

\begin{table}[ht]
        \caption{Términos de búsqueda} 
        \centering 
        \begin{tabular}{c c}
                \hline
                Concepto & Término de búsqueda\\ [0.5ex] % inserts table
                %heading
                \hline
                Requisitos             & \makecell{Requirements Engineering \\
                                                   Collaborative Requirements} \\
                \hline 
                Software como servicio & \makecell{Software as a Service \\
                                                   SaaS \\
                                                   Cloud Computing }\\ [1ex]
                \hline 
        \end{tabular}
        \label{table:tablaterminos}
\end{table}
\newpage

\subsection{Cadenas de búsqueda}
Basado en la estructura de las preguntas de investigación, se extrajo un conjunto de 
términos de búsqueda, éstos serán utilizados con el objetivo de definir una cadena de búsqueda apropiada para 
las necesidades de la RSL. 
En el proceso de llevar a cabo búsquedas piloto, se utilizaron los siguientes terminos de interés extraídos de las preguntas de la investigación: 
\emph{Requirement* y Elicitation o Validation o Analysis o Management o Risk Management}, las búsquedas con estos 
términos resultaron en un grupo de estudios primarios que no eran de importancia para alcanzar el objetivo de la 
revisión sistemática. 

Según lo anterior se estableció la siguiente cadena.

\emph{(Requirement* Engineering OR Collaborative Engineering AND (Software as a service or Saa* or Cloud computing))}

Según la fuente de datos a utilizar, se realizaron un conjunto de modificaciones para satisfacer las necesidades específicas de cada fuente de datos,


\subsection{Evaluación de búsqueda automatizada}
Para la validación de la cadena de búsqueda se tomó como referencia \cite{Evidence-Based}
para tener un punto de referencia que sea objetivo sobre el nivel de completitud para la RSL, 
esto, para asegurar que existe un número apropiado de estudios obtenidos en la búsqueda automatizada.
Los criterios clave para evaluar el nivel de completitud son \emph{Recall} y \emph{Precision}. \\ 
El \emph{Recall} de una búsqueda es la proporción de todos los estudios relevantes encontrados en una búsqueda. \\
El \emph{Precision} de una búsqueda es la proporción de los estudios encontrados que son relevantes a las preguntas 
planteadas en la investigación. 
Se va a calcular el \emph{Recall} y \emph{Precision} para cada cadeba de búsqueda utilizada en las fuentes de datos. 
\newpage


\subsection{Selección de fuentes}
Seleccionar bases de datos relevantes en el área de Tecnologías de la Información 
e Ingeniería de \emph{Software}, es fundamental para una revisión sistemática 
de la literatura.  Se seleccionaron las fuentes de información desplegadas en el Cuadro \ref{tablafuentes},
ya que disponen de acceso a trabajos sustanciales en los campos de ingeniería de requisitos y software como servicio, 
así como también a las conferencias y journals importantes. 
Antes de definir el conjunto de bases de datos, se llevaron a cabo 
búsquedas prueba, esto culminó en la exclusión  de \emph{Google Schoolar}, 
por el número de artículos repetidos.\\
Es importante notar que cada fuente de datos contiene un conjunto de opciones para búsquedas avanzadas, 
esto se tomó en cuenta para posteriormente, diseñar criterios individuales con el objetivo de mejorar la calidad de inclusión de los 
artículos de interés para el estudio.


\begin{table}[ht]
        \caption{Fuentes seleccionadas\strut}
        \label{tablafuentes} 
        \centering 
        \begin{tabular}{c}
                \hline
                Fuentes \\ 
                %heading
                \hline 
                \makecell{iEEE Explore\\
                          Science Direct\\
                          ACM Digital Library} \\ [1ex] 
                \hline 
        \end{tabular}
\end{table}
\newpage

\section{Selección de estudios primarios}
Se seleccionó búsqueda automatizada sugerida por \cite{kitchenham2007guidelines} para obtener el mayor número de 
artículos posibles, con un alto nivel de precisión.
Para validar el proceso de selección, se llevarán a cabo un conjunto de búsquedas 
informales en librerías indexadoras, fuentes digitales y conferencias influyentes en el área de 
servicios cloud, también búsquedas manuales para validar que los estudios base sean encontrados utilizando
el proceso de búsqueda, las fuentes, criterios de inclusión y exclusión y síntesis de los datos.

La completetitud de análisis de los estudios primarios se ha definido como importante ya que 
existe un conjunto limitado de conocimiento en el área. Para mitigar lo anterior, se ha añadido el proceso de 
validación de completitud.
El conocimiento adquirido en los estudios es de alta importancia, ya que las actividades 
internas del proceso de requisitos en un software como servicio puede variar a través de los estudios
seleccionados y es importante analizar un rango amplio de estudios que proyecten el estado actual del arte 
de requisitos de un \emph{software} como servicio. 

\subsection{Criterios de selección de estudios primarios}
Se definieron criterios de inclusión y exclusión con el objetivo 
de seleccionar investigaciones que respondan las preguntas de investigación, con la finalidad de ser  
sintetizadas para extraer su información al fin de la revisión de estudios.
Se incluyen solo estudios primarios escritos en inglés (CI-1) ya que no existe el
recurso humano para traducir estudios en otros idiomas, durante las búsquedas piloto 
se definió incluir estudios realizados entre 2010  y diciembre del 2020 (CI-2), ya que es importante
encontrar trabajos relevantes recientes relacionados al software como servicio, 
se excluye literatura informal (CE-1), estudios duplicados (CE-2), se incluyen 
estudios según el análisis de título y abstract (CI-3) y (CI-4),
se incluye si el texto completo contesta a alguna de las preguntas de investigación (CI-5), 
se excluye si es una versión previa a un estudio más completo (CE-3), o si no es posible acceder desde la fuente 
de información (CE-4).

\subsection{Criterios de inclusión}
\begin{enumerate}[C-1.-]
  \item{Es un estudio primario escrito en inglés.}
  \item{Es un estudio primario publicado entre 2010 - diciembre del 2021.}
  \item{El título y el abstract dan indicios de que se concentrará en una de las preguntas de investigación.}
  \item{El título y el abstract deben contener al menos dos términos de búsqueda.}
  \item{El texto completo contesta a alguna de las preguntas de investigación.}
\end{enumerate}

\subsection{Criterios de exclusión}
\begin{enumerate}[CE-1.-]
  \item{Es un libro, capítulo de libro, curso o estándar.}
  \item{Es un estudio primario duplicado. (Aparece en más de una base de datos.)}
  \item{Es una versión previa a un estudio más completo sobre la misma investigación.}
  \item{No se tiene acceso al texto completo.}
\end{enumerate}
\newpage

\subsection{Procedimiento de selección de estudios primarios}
Se definieron las siguientes etapas para la selección de estudios primarios con el fin 
de filtrar de forma eficaz la selección de estudios, para facilitar la selección y análisis 
del conjunto de estudios, con el fin de obtener una base de conocimientos relevantes. 

\subsection{Etapa número uno}
\begin{enumerate}[(a)]
  \item{Idioma inglés. (CI1)}
  \item{Publicado entre 2010-2020. (CI2)}
  \item{No es un libro, capítulo de libro, curso o estándar. (CE1)}
  \item{El título y abstract dan indicios de que se trata del dominio de interés. (CI4)}
\end{enumerate}

\subsection{Etapa número dos}
\begin{enumerate}[(a)]
  \item{Contiene al menos dos términos de búsqueda. (CI3)}
  \item{No duplicados. (CE2)}
  \item{No hay versiones anteriores. (CE3) }
  \item{Acceso al texto completo. (CE4)}
\end{enumerate}

\subsection{Etapa número tres}
\begin{enumerate}[(a)]
  \item{Texto completo contesta alguna pregunta de investigación. (CI5)}
\end{enumerate}
\newpage

\section{Evaluación de calidad}
Incluso si no existe un éstandar que establezca las caracteristicas de un estudio 
de alta calidad, hay un consenso común sobre el impacto de los estudios primarios 
en los resultados de una revisión sistemática de la literatura, por eso y utilizando 
preguntas definidas según \cite{Evidence-Based} con el fin de evaluar el proceso y los
resultados señalados en los estudios correspondientes. 

\begin{center}
\begin{table}[ht]
        \caption{Calidad} 
        \centering 
        \scalebox{0.6}{
        \begin{tabular}{c c c}
                \hline
                Criterios & Grado & Grado obtenido\\ [0.5ex]
                %heading
                \hline 
                (C1) ¿Es el objetivo del estudio definido de forma clara? & {1, 0.5, 0} {Si, Nominalmente, No} \\
                \hline
                (C2) ¿El contexto del estudio está bien definido? & {1, 0.5, 0} {Si, Nominalmente, No} \\
                \hline 
                (C3) ¿Los resultados son claros? & {1, 0.5, 0} {Si, Nominalmente, No} \\
                \hline
                (C4) Según los resultados, ¿Que tan valioso es el estudio? & {1, 0.5, 0} {Altamente, Neutral, No aporta valor}\\
                \hline 
        \end{tabular}}
        \label{table:tablacalidad}
\end{table}
\end{center}

\newpage

\section{Extracción de los datos}
\subsection{Formato para extracción de los datos}
Para llevar a cabo la extracción de los datos, se definió un formato de extracción compuesto por dos partes; datos generales y contexto.
En la sección de datos generales se guardará la información importante relacionada a la publicación, con el propósito de guardar 
una referencia bibligráfica del estudio, así como identificar año de publicación y palabras clave relacionadas a la investigación. 

\begin{table}[h!]
    \begin{center}
    \caption{Tabla de información general.}
    \label{tab:datosgenerales}
    \begin{tabular}{|l|c|} 
    \hline
    \multicolumn{2}{|c|}{Información general} \\
    \hline
    Identificador &       \\
    \hline
    Título & \\
    \hline
    Autores &\\
    \hline
    Daño & \\
    \hline
    Fuente & \\
    \hline
    Título de publicación (memorias, \emph{journal}, etc.) & \\
    \hline
    DOI & \\
    \hline
    Palabras clave & \\
    \hline
    \emph{Abstract} o resumen & \\
    \hline
    \end{tabular}
    \end{center}
\end{table}

En la sección de contexto, se almacena la información fundamental para la revisión sistemática, ya que contiene los datos que responderán a las
preguntas de investigación, para lograr lo anterior se definió un grupo de campos para identificar la pregunta o preguntas posibles a ser contestadas 
por el estudio, el primer campo definido, se utiliza para identificar las técnicas de elicitación, que es una respuesta a la PI1, también se creó un campo para 
documentar los retos presentados en la elicitación que contesta a la sub-pregunta PI1-A, existe otro campo para almacenar las técnicas utilizadas para llevar a cabo 
el análisis de requisitos, que da respuesta a la PI2, después se especifica otro campo para albergar las actividades realizadas para realizar la validación de requisitos que 
dará respuesta a la pregunta PI3, por último se crea un campo para almacenar los temas abiertos propuestos observados en el estudio, que servirá para responder la pregunta PI4.


\begin{table}[h!]
    \begin{center}
    \caption{Tabla de contexto.}
    \label{tab:datosgenerales}
    \begin{tabular}{|l|c|}
    \hline
    \multicolumn{2}{|c|}{Contexto} \\
    \hline
    Pregunta/s de investigación relacionada/s & \\
    \hline
    Técnica/s identificada/s para elicitación& \\
    \hline
    Reto/s identificado/s en el uso de técnicas de elicitación& \\
    \hline
    Técnica/s de análisis de requisitos identificadas& \\
    \hline
    Técnica/s utilizada/s para validar requisitos & \\
    \hline
    Tema/s abierto/s propuesto/s en el área de requisitos en el software como servicio& \\
    \hline
    \end{tabular}
    \end{center}
\end{table}


\section{Estrategia para la síntesis de datos}
El proceso de síntesis se llevará a cabo de forma narrativa, para el cual se escogió como guía para conducción de síntesis narrativa (referencia). 
La primera fase “Desarrollar una teoría” nos permite contribuir a la interpretación de la información recopilada y evalúa qué tan aplicable es la 
información en el contexto de la revisión. En la segunda fase “Desarrollar una síntesis preliminar” se realiza un análisis general de los datos 
para presentar una síntesis preliminar, con el objetivo de identificar patrones o relaciones entre la información recabada, en la tercera fase 
“Explorar la relación entre los datos.” se repite el análisis sobre la información obtenida para entender la razón de las relaciones y sus efectos 
en los resultados del estudio. Siguiendo con la cuarta fase “Evaluar la robustez del producto de la síntesis” se habla sobre seguir un proceso 
para evaluar la calidad de los estudios obtenidos, ya que la robustez de la síntesis depende de la calidad metodológica de los estudios, con el 
propósito de definir un proceso de evaluación orientado a robustez, se definieron los criterios de calidad con una perspectiva hacia el método. 
En la guía no existe una última fase definida, sin embargo se agregó la fase “Conclusiones y recomendaciones” 
con el fin de identificar e informar los hallazgos encontrados en la investigación, así como las áreas de oportunidad encontradas..  


\begin{figure}[!htb]
   \includegraphics[width=\linewidth]{narrativa.png}
   \caption{Proceso de síntesis narrativa.}
   \label{fig:sistesisnarrativa}
\end{figure}

\newpage

\section{Limitaciones}
\subsection{Amenazas a la validez: Internas}
El factor principal a ser considerado como restricción es que el estudio presente está siendo 
desarrollado por un estudiante de licenciatura, para minimizar esta amenaza, que puede afectar a la objetividad 
al realizar la RSL se tomó como referencia la guía realizada por \cite{kitchenham2007guidelines}
como guía para llevar a cabo la RSL, también se definieron evaluaciones de la calidad del trabajo, en conjunto 
con los directores de la revisión para mitigar cualquier amenaza a la objetividad de la investigación. 

\subsection{Amenazas a la validez: Externas}
El campo de software como servicio es nuevo y sigue en crecimiento, esto se hace notorio en la escasez de 
RSL relacionadas al tema, por lo tanto se define lo anterior como  una amenaza externa ya que, un conjunto 
de revisiones sistemáticas podrían formar una base de fundamentos para el desarrollo de temas de investigación, el cual no existe.
\newpage

\section{Informe}
\subsection{Titulo}
\subsection{Resumen}
\subsection{Introducción}
\subsection{Antecedentes}
\subsection{Método}
\subsection{Resultados}
\subsection{Discusión}
\subsection{Conclusión}
\newpage

\section{Gestión de la revisión}

\subsection{Cronograma}
El cronograma muestra las actividades a llevar a cabo en los meses de noviembre a enero.

\begin{figure}[!htb]
    \includegraphics[width=\linewidth]{gant.png}
    \caption{Cronograma de actividades.}
    \label{fig:grant}
\end{figure}

\subsection{Herramientas utilizadas}
Se utilizó LaTex para llevar a cabo el documento del protocolo, para almacenar referencias 
se crearon diferentes archivos, los cuales contienen referencias, todos han sido 
almacenados con extensión .bib, se empleó BibTex para llevar a cabo las referencias a las 
investigaciones o documentos de interés, guardó la hoja de cálculo en la que se llevó 
a cabo búsquedas prueba y se se gestionó el proceso de control del documento con .git,
los archivos de referencias han sido alojados en GitHub,
\newpage

\subsubsection{Paquete de replicación}
Como un aporte a la comunidad de investigación en el área de ingeniería de \emph{software}, 
se añade un paquete de replicación de la investigación, con el objetivo de aportar  
al conjunto de estudiantes e investigadores interesados y de esta forma avanzar y distribuir 
conocimiento.
El paquete de replicación también será una herramienta importante utilizada con el fin de replicar 
el proceso de investigación llevado a cabo.

Lista de items incluídos en el paquete: 
  \begin{enumerate}[(a)]
  \item Anteproyecto con descripción de objetivo y contexto de la investigación. 
  \item Lista de referencias utilizadas como fundamento de conocimiento. (.bib)
  \item Lista de referencias utilizadas en el protocolo.  (.bib)
  \item Protocolo.
  \item Tabla de \emph{Google sheets} con detalle de búsquedas piloto realizadas. 
  \item Lista de referencias utilizadas en el protocolo de investigación.
  \end{enumerate}
\newpage

\section{Referencias}
\printbibliography

\end{document}
