\documentclass{article}

\usepackage{enumerate}
\usepackage{makecell}
\usepackage[spanish]{babel}
\usepackage[utf8]{inputenc}

\begin{document}
  \title{%
  Protocolo \\
  \large Revisión de la literatura sobre las actividades de requisitos para Software como Servicio\\}
  \author{Alberto de Jesús Sánchez López \\ 
  \small Proyecto Guiado}
  \date{Fecha}
  \maketitle
  \thispagestyle{empty}
  \newpage

  \tableofcontents
  \thispagestyle{empty}
  \newpage

\setcounter{page}{1}
\section{Introducción}
Llevar a cabo un proceso de desarrollo de software que resulte en un producto de calidad, es 
un proceso complejo, que utiliza metodologías y estrategias adecuadas para el análisis del problema 
en cuestión asì cómo el diseño de una solución, para el Software como Servicio (SaaS), esta no es la excepción. 
El Software como Servicio tiene necesidades específicas que evitan la adaptación de metodologías y estrategias 
tradicionales utilizadas dentro de la Ingeniería de Software, esto representa un conjunto de problemas para aquellos 
interesados en iniciar el desarrollo de un software como servicio.

El proceso de desarrollo de software tradicional conlleva un conjunto de fases con el propósito común de asegurar la entrega 
de un producto con calidad y en tiempo, la fase de requisitos, es crucial para delimitar el alcance del proyecto, analizar, 
documentar y verificar los servicios y restricciones del sistema. Una definición de requisitos que ha sido desarrollada siguiendo una 
metodogìa o estrategia, es de suma importancia para el éxito del proyecto, ya que esto asegura que la especificación de requisitos ha sido 
desarrollada siguiendo un proceso formal y por ende los fundamentos necesarios para el correcto diseño de la solución son confiables.

Es por eso, que es de especial interés para aquellos involucrados en el desarrollo de un proceso que resulte en un producto de software 
con calidad. Una fase crucial para el éxito de un proyecto de software es la definición de requisitos. 
\newpage

\section{Preguntas de investigación}
Con el 

\begin{enumerate}[P 1.-]
  \item\emph{¿Qué técnicas de elicitación se han utilizado para la identificación de requisitos de Software como Servicio? }
  \begin{enumerate}[(a)]
  \item \emph{¿Qué fuentes de información son utilizadas?}
  \item \emph{¿Qué técnicas de elicitación se aplican?}
  \item \emph{¿Qué retos se presentan en la elicitación?}
  \end{enumerate}
  
  \item\emph{¿Qué técnicas de análisis se han utilizado para la definición de requisitos de Software como Servicio?}
  \begin{enumerate}[(a)]
  \item \emph{¿Cómo se clasifica la información proveniente de la elicitación?}
  \item \emph{¿Cómo se definen los requisitos a partir de la información organizada?}
  \item \emph{¿Cómo se identifican los requisitos no funcionales?}
  \end{enumerate}

  \item\emph{¿Cómo se define el proceso de validación de los requisitos de un Software como servicio?}
  \begin{enumerate}[(a)]
  \item \emph{¿Qué técnicas se utilizan para la validación de los requisitos? }
  \end{enumerate}

  \item\emph{¿Cómo se gestionan los cambios de los requisitos de un Software como Servicio?}
  \begin{enumerate}[(a)]
  \item \emph{¿Cómo se monitorea el entorno para identificar cambios en las condiciones de los requisitos?}
  \item \emph{¿Cómo se evalúan los cambios en los requisitos?}
  \end{enumerate}
            
  \item\emph{¿Cómo se gestionan los riesgos de los requisitos de un Software como Servicio?}
  \begin{enumerate}[(a)]
  \item \emph{¿Cómo se identifican los riesgos?}
  \item \emph{¿Cómo se evalúan los riesgos?}
  \item \emph{¿Cómo se monitorean los riesgos?}
  \item \emph{¿Cómo se controlan los riesgos?}
  \end{enumerate}
  
  \item\emph{¿Qué temas abiertos se identifican en la literatura reciente en el desarrollo de software como servicio?}
  \begin{enumerate}[(a)]
  \item \emph{¿Qué temas abiertos existen relacionado a métodos para requisitos?}
  \item \emph{¿Qué temas abiertos existen relacionado a herramientas para requisitos?}
  \end{enumerate}

\end{enumerate}

\newpage


\section{Estrategia de búsqueda}


\subsection{Términos de búsqueda}
Los siguientes términos de búsqueda fueron seleccionados con el propósito de identificar los estudios que 
permiten proveer evidencia que es relevante a las preguntas de investigación

\begin{table}[ht]
        \caption{Términos de búsqueda} 
        \centering 
        \begin{tabular}{c c}
                \hline
                Concepto & Término de búsqueda\\ [0.5ex] % inserts table
                %heading
                \hline
                Requisitos             & \makecell{Requirements Engineering \\
                                                   Requirements Elicitation \\
                                                   Requirements Validation \\
                                                   Requirements Analysis \\
                                                   Requirements Management \\
                                                   Requirements Risk Management} \\ 
                \hline 
                Software como servicio & \makecell{Software as a Service \\
                                                   SaaS \\
                                                   Cloud Computing \\
                                                   Cloud Services\\
                                                   Service-oriented Requirements Engineering \\
                                                   SORE } \\[1ex] 
                \hline 
        \end{tabular}
        \label{table:tablaterminos}
\end{table}
\newpage

\subsection{Cadenas de búsqueda}

\subsection{Selección de fuentes}
Seleccionar bases de datos relevantes en el área de Tecnologías de la Información 
e Ingeniería de \emph{Software}, es fundamental para una revisión sistemática 
de la literatura.  Se seleccionaron las fuentes de información desplegadas en el Cuadro \ref{tablafuentes},
ya que dispone de acceso a trabajos sustanciales en ingeniería de requisitos y software como servicio, 
así como también a las conferencias y journals importantes en ambos campos. 
Antes de definir el conjunto de bases de datos ya mencionados, se llevaron a cabo un conjunto de decisiones 
motivadas por la ejecución de búsquedas prueba, esto culminó en la exclusión  de \emph{Google Schoolar}, 
por el número de artículos repetidos.\\
Es importante notar que cada fuente de datos contiene un conjunto de opciones para búsquedas avanzadas, 
esto se tomó en cuenta para posteriormente, diseñar criterios individuales con el objetivo de mejorar la calidad de inclusión de los 
artículos de interés para el estudio.


\begin{table}[ht]
        \caption{Fuentes seleccionadas\strut}
        \label{tablafuentes} 
        \centering 
        \begin{tabular}{c}
                \hline
                Fuentes \\ 
                %heading
                \hline 
                \makecell{iEEE Explore\\
                          Science Direct\\
                          ACM Digital Library} \\ [1ex] 
                \hline 
        \end{tabular}
\end{table}
\newpage

\section{Selección de estudios primarios}



\subsection{Criterios de selección de estudios primarios}
Se definieron criterios de inclusión y exclusión con el objetivo 
de seleccionar investigaciones relevantes para el análisis y posteriormente, 
la síntesis de información.
Se incluyen solo estudios primarios escritos en inglés (CI-1) ya que no existe el
recurso humano para traducir estudios en otros idiomas, durante las búsquedas piloto 
se definió incluir estudios realizados entre 2011 y marzo del 2020, ya que no se encontró evidencia 
relevante de trabajos relacionados al software como servicio antes, se excluye literatura informal (CE-1),
estudios duplicados (CE-2), se incluyen estudios según el análisis de título y abstract(CI-3) y (CI-4),
se incluye si el texto completo contesta a alguna de las preguntas de investigación (CI-5), 
se excluye si es una versión previa a un estudio más completo (CE-3), o si no es posible acceder desde la fuente 
de información (CE-4).


\subsection{Criterios de inclusión}
\begin{enumerate}[C-1.-]
  \item{Es un estudio primario escrito en inglés.}
  \item{Es un estudio primario publicado entre 2015 - marzo del 2020}
  \item{El título y el abstract dan indicios de que se concentrará en una de las preguntas de investigación.}
  \item{El título y el abstract deben contener al menos dos términos de búsqueda.}
  \item{El texto completo contesta a alguna de las preguntas de investigación.}
\end{enumerate}

\subsection{Criterios de exclusión}
\begin{enumerate}[CE-1.-]
  \item{Es un libro, capítulo de libro, curso o estándar}
  \item{Es un estudio primario duplicado. (Aparece en más de una base de datos)}
  \item{Es una versión previa a un estudio más completo sobre la misma investigación.}
  \item{No se tiene acceso al texto completo.}
\end{enumerate}
\newpage

\subsection{Procedimiento de selección de estudios primarios}

\subsection{Etapa número uno}
\begin{enumerate}[(a)]
  \item{Idioma inglés. (CI1)}
  \item{Publicado entre 2015-2020. (CI2)}
  \item{No es un libro, capítulo de libro, curso o estándar. (CE1)}
\end{enumerate}

\subsection{Etapa número dos}
\begin{enumerate}[(a)]
  \item{El título y abstract dan indicios de que se trata del dominio de interés. (CI3)}
  \item{No duplicados. (CE2)}
  \item{No hay versiones anteriores. (CE3) }
  \item{Acceso al texto. (CE4)}
\end{enumerate}

\subsection{Etapa número tres}
\begin{enumerate}[(a)]
  \item{Contiene al menos dos términos de búsqueda. (CI4)}
  \item{Texto completo contesta alguna pregunta de investigación. (CI5)}
\end{enumerate}
\newpage

\section{Evaluación de calidad}
Incluso si no existe un éstandar que establezca las caracteristicas de un estudio 
de alta calidad, hay un consenso común sobre el impacto de los estudios primarios 
en los resultados de una revisión sistemática de la literatura, por eso. 


\begin{table}[ht]
        \caption{Términos de búsqueda} 
        \centering 
        \begin{tabular}{c c c}
                \hline
                Criterios & Grado & Grado obtenido\\ [0.5ex]
                %heading
                \hline 
                (C1) ¿Es el objetivo del estudio definido de forma clara? & {1, 0.5, 0} {Si, Nominalmente, No} \\
                \hline
                (C2) ¿El contexto del estudio está bien definido? & {1, 0.5, 0} {Si, Nominalmente, No} \\
                \hline 
                (C3) ¿Los resultados son claros? & {1, 0.5, 0} {Si, Nominalmente, No} \\
                \hline
                (C4) Según los resultados, ¿Que tan valioso es el estudio? \\
                \hline 
        \end{tabular}
        \label{table:tablaterminos}
\end{table}

\subsection{Procedimiento de aplicación de evaluación}
\section{Extracción de los datos}
\subsection{Formato para extracción de los tados}
\section{Estrategia para la síntesis de datos}
\section{Limitaciones}
\subsection{Amenazas a la validez: Internas}
\subsection{Amenazas a la validez: Externas}
\section{Informe}
\subsection{Titulo}
\subsection{Resumen}
\subsection{Introducción}
\subsection{Antecedentes}
\subsection{Método}
\subsection{Resultados}
\subsection{Discusión}
\subsection{Conclusión}
\section{Gestión de la revisión}
\section{Referencias}

\end{document}
